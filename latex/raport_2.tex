%╔════════════════════════════╗
%║      Szablon wykonał       ║
%║  mgr inż. Dawid Kotlarski  ║
%║        06.10.2024          ║
%╚════════════════════════════╝

\documentclass[12pt,a4paper]{mwart}
\usepackage[utf8]{inputenc}
\usepackage{polski}
\usepackage[T1]{fontenc}
\usepackage{amsmath}
\usepackage{amsfonts}
\usepackage{amssymb}
\usepackage{graphicx}
\usepackage{array}
\usepackage{multirow}
\usepackage{geometry}
\usepackage{tabularray}

% Ustawienie legalpaper jest bardzo długie, jeśli drukujesz na A4, 
% treść i tak się zmieści bez problemu.
\geometry{legalpaper, margin=1.5cm}

\renewcommand{\arraystretch}{1.2}

\begin{document}
    
\begin{center}
    \Huge RAPORT
\end{center}

\begin{table}[h!]
    \centering
    \begin{tblr}{
            width = \linewidth,
            colspec = {Q[156]Q[156]Q[156]Q[156]Q[156]Q[156]},
            row{1} = {c},
            column{4} = {c},
            column{6} = {c},
            cell{1}{1} = {c=6}{0.936\linewidth},
            cell{2}{2} = {c=5}{0.803\linewidth},
            cell{3}{2} = {c=5}{0.803\linewidth},
            cell{4}{2} = {c},
            cell{5}{2} = {c=5}{0.803\linewidth},
            hline{1,6} = {1}{-}{leftpos = 1, rightpos = 1},
            hline{1,6} = {2}{-}{leftpos = 1, rightpos = 1},
            hline{2,2} = {1}{-}{leftpos = 1, rightpos = 1},
            hline{2,2} = {2}{-}{leftpos = 1, rightpos = 1},
            vline{1,1} = {1}{-}{abovepos = 1, belowpos = 1},
            vline{1,1} = {2}{-}{abovepos = 1, belowpos = 1},
            vline{7,1} = {1}{-}{abovepos = 1, belowpos = 1},
            vline{7,1} = {2}{-}{abovepos = 1, belowpos = 1},
            hlines,
            vlines,
        }
        {AKADEMIA NAUK STOSOWANYCH W NOWYM SĄCZU\\Wydział Nauk Inżynieryjnych, Katedra informatyki} &  &  &  &  &  \\
        Przedmiot:  & Programowanie urządzeń mobilnych – projekt, mgr inż. Dawid Kotlarski          &  &  &  &  \\
        Temat:      & Implementacja mapy i bazy danych                                              &  &  &  &  \\
        Grupa:      & IS-2(n)P1  & Nr raportu: & 2 & Data: & 16.01.2026 \\
        Osoby:      & Rafał Curzydło,  Dominik Jonik                                                &  &  &  &             
    \end{tblr}
\end{table}

\vspace{-0.5cm} % Lekkie podciągnięcie treści pod tabelę

\section*{Wykonane zadania}
\begin{itemize}
    \item \textbf{Mapy i UI:} Zintegrowano bibliotekę \textbf{MapBox} w zakładce MAPA oraz dodano przycisk centrowania widoku na użytkowniku.
    \item \textbf{Sensory:} Przywrócono i skonfigurowano obsługę akcelerometru.
    \item \textbf{Baza danych (Room):} Zaimplementowano warstwę persystencji danych z wykorzystaniem biblioteki Room (SQLite).
    \begin{itemize}
        \item Utworzono encję \texttt{DailySummary} (poziom ogólny) z kluczem w formacie \texttt{yyyy-MM-dd} do agregacji kroków.
        \item Utworzono encję \texttt{Workout} (poziom szczegółowy) zawierającą znaczniki czasu (\texttt{startTime}, \texttt{endTime}) oraz dane trasy (\texttt{routePointsJson}).
        \item Zdefiniowano relacje z kluczem obcym i opcją \texttt{CASCADE} (usuwanie dnia usuwa powiązane treningi).
    \end{itemize}
\end{itemize}

\section*{Niewykonane zadania}
\begin{itemize}
    \item \textbf{Wizualizacja GPS:} Brak implementacji renderowania przebytej trasy na warstwie mapy (dane są zbierane, ale niewidoczne graficznie).
\end{itemize}

\section*{Napotkane problemy}
\begin{itemize}
    \item \textbf{Wydajność bazy danych:} Implementacja zapisu danych powodowała blokowanie wątku głównego (UI), co skutkowało spadkiem płynności aplikacji. Wymagane przeniesienie operacji I/O do wątków tła.
    \item \textbf{Obsługa danych GPS:} Problemy koncepcyjne ze sposobem serializacji i optymalnego zapisu dużej liczby punktów trasy do bazy danych.
\end{itemize}

\section*{Zadania na kolejny tydzień}
\begin{itemize}
    \item Pełna implementacja obsługi GPS: pobieranie lokalizacji, zapis trasy do bazy oraz jej wizualizacja na mapie.
    \item Eliminacja błędów i optymalizacja wydajności (rozwiązanie problemów z przycinaniem interfejsu).
    \item Przystosowanie aplikacji do wdrożenia (przygotowanie wersji produkcyjnej).
\end{itemize}

\section*{Stan dokumentacji}
Dokumentacja została zaktualizowana i zawiera 3 w pełni kompletne rozdziały:
\begin{itemize}
    \item \textbf{Rozdział 1:} Ogólne określenie wymagań.
    \item \textbf{Rozdział 2:} Określenie wymagań szczegółowych.
    \item \textbf{Rozdział 3:} Projektowanie (w tym opisany schemat i zasada działania bazy danych).
\end{itemize}

\end{document}
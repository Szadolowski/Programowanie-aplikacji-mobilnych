
%╔════════════════════════════╗
%║		Szablon wykonał		  ║
%║	mgr inż. Dawid Kotlarski  ║
%║		  06.10.2024		  ║
%╚════════════════════════════╝

\documentclass[12pt,a4paper]{mwart}
\usepackage[utf8]{inputenc}
\usepackage{polski}
\usepackage[T1]{fontenc}
\usepackage{amsmath}
\usepackage{amsfonts}
\usepackage{amssymb}
\usepackage{graphicx}
\usepackage{array}
\usepackage{multirow}
\usepackage{geometry}
\usepackage{tabularray}

\geometry{legalpaper, margin=1.5cm}

\renewcommand{\arraystretch}{1.2}

\begin{document}
	
\begin{center}
	\Huge RAPORT
\end{center}

\begin{table}[h!]
	\centering
	\begin{tblr}{
			width = \linewidth,
			colspec = {Q[156]Q[156]Q[156]Q[156]Q[156]Q[156]},
			row{1} = {c},
			column{4} = {c},
			column{6} = {c},
			cell{1}{1} = {c=6}{0.936\linewidth},
			cell{2}{2} = {c=5}{0.803\linewidth},
			cell{3}{2} = {c=5}{0.803\linewidth},
			cell{4}{2} = {c},
			cell{5}{2} = {c=5}{0.803\linewidth},
			hline{1,6} = {1}{-}{leftpos = 1, rightpos = 1},
			hline{1,6} = {2}{-}{leftpos = 1, rightpos = 1},
			hline{2,2} = {1}{-}{leftpos = 1, rightpos = 1},
			hline{2,2} = {2}{-}{leftpos = 1, rightpos = 1},
			vline{1,1} = {1}{-}{abovepos = 1, belowpos = 1},
			vline{1,1} = {2}{-}{abovepos = 1, belowpos = 1},
			vline{7,1} = {1}{-}{abovepos = 1, belowpos = 1},
			vline{7,1} = {2}{-}{abovepos = 1, belowpos = 1},
			hlines,
			vlines,
		}
		{AKADEMIA NAUK STOSOWANYCH W NOWYM SĄCZU\\Wydział Nauk Inżynieryjnych, Katedra informatyki} &  &  &  &  &  \\
		Przedmiot:  & Programowanie urządzeń mobilnych – projekt, mgr inż. Dawid Kotlarski          &  &  &  &  \\
		Temat:      & Początek projektu                                                          &  &  &  &  \\
		Grupa:      & IS-2(s)P1  & Nr raportu: & 1 & Data: & 09.01.2026 \\
		Osoby:      & Rafał Curzydło,  Dominik Jonik                                          &  &  &  &            
	\end{tblr}
\end{table}


\section{Wykonane zadania}

W bieżącym sprincie prace koncentrowały się na budowie interfejsu użytkownika oraz logiki sensorów, a także na opracowaniu wymagań w dokumentacji.

\begin{itemize}
    \item \textbf{Implementacja Interfejsu (UI):}
    \begin{itemize}
        \item Zbudowano kompletny szkielet aplikacji oparty na Fragmentach i \texttt{BottomNavigationView}.
        \item Przygotowano widok "Historia" z wykorzystaniem \texttt{RecyclerView}. W celu weryfikacji czytelności interfejsu zaimplementowano wyświetlanie przykładowych danych testowych (tzw. \textit{hardcoded mock data}).
        \item Zaktualizowano stylistykę do Material Design 3, dodając dedykowany przycisk sterujący (FAB) na dolnym pasku.
    \end{itemize}

    \item \textbf{Logika Sensorów:}
    \begin{itemize}
        \item Zgodnie z założeniami dokumentacji[cite: 59], zaimplementowano obsługę sprzętowego licznika kroków (\texttt{TYPE\_STEP\_COUNTER}), rezygnując z mniej dokładnego akcelerometru.
        \item Dodano obsługę uprawnień systemowych niezbędnych do działania w tle.
    \end{itemize}

    \item \textbf{Dokumentacja:}
    \begin{itemize}
        \item Opracowano rozdział "Ogólne określenie wymagań"[cite: 16], definiujący cele biznesowe i funkcjonalne.
        \item Sporządzono specyfikację techniczną w rozdziale "Określenie wymagań szczegółowych"[cite: 49].
    \end{itemize}
\end{itemize}

\newpage

% Sekcja ze zrzutami ekranu (nazwy plików zgodnie z ustaleniami)
\begin{figure}[h!]
    \centering
    \begin{minipage}{0.45\textwidth}
        \centering
        \includegraphics[width=0.9\textwidth]{rys/tracker.jpg}
        \caption{Ekran główny (Licznik)}
    \end{minipage}\hfill
    \begin{minipage}{0.45\textwidth}
        \centering
        \includegraphics[width=0.9\textwidth]{rys/historia.jpg}
        \caption{Widok Historii (dane testowe)}
    \end{minipage}
\end{figure}

\begin{figure}[h!]
    \centering
    \begin{minipage}{0.45\textwidth}
        \centering
        \includegraphics[width=0.9\textwidth]{rys/profil.jpg}
        \caption{Profil użytkownika}
    \end{minipage}\hfill
    \begin{minipage}{0.45\textwidth}
        \centering
        \includegraphics[width=0.9\textwidth]{rys/mapa.jpg}
        \caption{Widok mapy (w przygotowaniu)}
    \end{minipage}
\end{figure}


\section{Niewykonane zadania}

Ze względu na konieczność głębszej analizy technologicznej, przesunięto realizację następujących elementów:
\begin{itemize}
    \item \textbf{Persystencja danych:} Nie zaimplementowano jeszcze docelowej bazy danych SQLite. Obecnie aplikacja nie zapamiętuje trwale historii po zamknięciu procesu – dane w widoku historii są symulowane.
    \item \textbf{Moduł Mapy:} Nie wybrano ostatecznego dostawcy map (Google Maps SDK vs OpenStreetMap). Widok mapy jest przygotowany w nawigacji, ale nie wyświetla jeszcze podkładu kartograficznego ani trasy.
\end{itemize}

\section{Napotkane problemy}

Podczas prac projektowych zidentyfikowano kluczowe problemy do rozwiązania:
\begin{itemize}
    \item \textbf{Wybór technologii mapowej:} Zespół napotkał problem decyzyjny dotyczący doboru SDK do obsługi map. Należy przeanalizować, czy darmowy OpenStreetMap spełni wymagania dotyczące rysowania trasy, czy konieczne będzie użycie komercyjnego Google Maps SDK (co wiąże się z konfiguracją kluczy API).
    \item \textbf{Struktura bazy danych:} Wyzwaniem jest zaprojektowanie optymalnego schematu tabel w SQLite, który pozwoli na wydajne zapisywanie tysięcy punktów współrzędnych GPS jednej trasy bez spowalniania działania aplikacji na starszych telefonach.
\end{itemize}

\section{Zadania na kolejny tydzień}

Plany na najbliższy sprint obejmują:
\begin{itemize}
    \item Przeprowadzenie researchu i podjęcie ostatecznej decyzji w sprawie biblioteki mapowej.
    \item Implementacja silnika bazy danych SQLite (stworzenie klasy \texttt{SQLiteOpenHelper} oraz kontraktów tabel).
    \item Zastąpienie danych "hardcoded" w historii danymi pobieranymi dynamicznie z nowo utworzonej bazy.
\end{itemize}

\section{Stan dokumentacji projektowej}

Dokumentacja została znacząco rozbudowana i przesłana w formie pliku PDF. Kluczowe zmiany obejmują:
\begin{itemize}
    \item Dodanie szczegółowego opisu założeń funkcjonalnych (Licznik, Tryb Treningowy, Dziennik)[cite: 22].
    \item Zdefiniowanie koncepcji technicznej opartej na lokalnej bazie SQLite oraz sensorach sprzętowych[cite: 52].
    \item Określenie wytycznych UX/UI, kładących nacisk na wysoki kontrast i czytelność w warunkach terenowych[cite: 41].
\end{itemize}

\end{document}
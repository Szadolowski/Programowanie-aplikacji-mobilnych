%╔════════════════════════════╗
%║		Szablon wykonał		  ║
%║	mgr inż. Dawid Kotlarski  ║
%║		  06.10.2024		  ║
%╚════════════════════════════╝

\documentclass[12pt,a4paper]{mwart}
\usepackage[utf8]{inputenc}
\usepackage{polski}
\usepackage[T1]{fontenc}
\usepackage{amsmath}
\usepackage{amsfonts}
\usepackage{amssymb}
\usepackage{graphicx}
\usepackage{array}
\usepackage{multirow}
\usepackage{geometry}
\usepackage{tabularray}

\geometry{legalpaper, margin=1.5cm}

\renewcommand{\arraystretch}{1.2}

\begin{document}
	
\begin{center}
	\Huge RAPORT
\end{center}

\begin{table}[h!]
	\centering
	\begin{tblr}{
			width = \linewidth,
			colspec = {Q[156]Q[156]Q[156]Q[156]Q[156]Q[156]},
			row{1} = {c},
			column{4} = {c},
			column{6} = {c},
			cell{1}{1} = {c=6}{0.936\linewidth},
			cell{2}{2} = {c=5}{0.803\linewidth},
			cell{3}{2} = {c=5}{0.803\linewidth},
			cell{4}{2} = {c},
			cell{5}{2} = {c=5}{0.803\linewidth},
			hline{1,6} = {1}{-}{leftpos = 1, rightpos = 1},
			hline{1,6} = {2}{-}{leftpos = 1, rightpos = 1},
			hline{2,2} = {1}{-}{leftpos = 1, rightpos = 1},
			hline{2,2} = {2}{-}{leftpos = 1, rightpos = 1},
			vline{1,1} = {1}{-}{abovepos = 1, belowpos = 1},
			vline{1,1} = {2}{-}{abovepos = 1, belowpos = 1},
			vline{7,1} = {1}{-}{abovepos = 1, belowpos = 1},
			vline{7,1} = {2}{-}{abovepos = 1, belowpos = 1},
			hlines,
			vlines,
		}
		{AKADEMIA NAUK STOSOWANYCH W NOWYM SĄCZU\\Wydział Nauk Inżynieryjnych, Katedra informatyki} &  &  &  &  &  \\
		Przedmiot:  & Programowanie urządzeń mobilnych – projekt, mgr inż. Dawid Kotlarski          &  &  &  &  \\
		Temat:      & Początek projektu                                                          &  &  &  &  \\
		Grupa:      & IS-2(n)P1  & Nr raportu: & 1 & Data: & 09.01.2026 \\
		Osoby:      & Rafał Curzydło,  Dominik Jonik                                          &  &  &  &            
	\end{tblr}
\end{table}

\vspace{-0.5cm} % Lekkie podciągnięcie treści pod tabelę

\section*{Wykonane zadania}
W bieżącym sprincie prace koncentrowały się na budowie interfejsu użytkownika oraz logiki sensorów, a także na opracowaniu wymagań w dokumentacji.

\begin{itemize}
    \item \textbf{Interfejs (UI):} Zbudowano szkielet (Fragmenty, \texttt{BottomNavigationView}) oraz widok Historii
    
    (\texttt{RecyclerView}) z danymi testowymi. Wdrożono styl Material Design 3 z przyciskiem FAB.
    \item \textbf{Logika Sensorów:} Zaimplementowano obsługę sprzętowego licznika kroków (\texttt{TYPE\_STEP\_COUNTER}) oraz wymagane uprawnienia systemowe.
    \item \textbf{Dokumentacja:} Opracowano cele biznesowe i specyfikację techniczną (wymagania ogólne i szczegółowe).
\end{itemize}

\section*{Niewykonane zadania}
Ze względu na konieczność analizy technologicznej przesunięto:
\begin{itemize}
    \item \textbf{Persystencja:} Brak bazy SQLite. Dane w historii są symulowane i nie są zapamiętywane po zamknięciu procesu.
    \item \textbf{Mapy:} Nie wybrano ostatecznego dostawcy (Google vs OSM). Widok mapy jest gotowy w nawigacji, ale nie wyświetla podkładu ani trasy.
\end{itemize}

\section*{Napotkane problemy}
\begin{itemize}
    \item \textbf{Technologia map:} Decyzja między darmowym OpenStreetMap a komercyjnym Google Maps SDK (wydajność rysowania trasy vs klucze API).
    \item \textbf{Baza danych:} Zaprojektowanie schematu SQLite, który wydajnie obsłuży zapis tysięcy punktów GPS bez spowalniania UI na starszych urządzeniach.
\end{itemize}

\section*{Zadania na kolejny tydzień}
\begin{itemize}
    \item Wybór i implementacja biblioteki mapowej.
    \item Stworzenie silnika bazy danych SQLite (klasa \texttt{SQLiteOpenHelper}, kontrakty).
    \item Zastąpienie danych "hardcoded" w historii danymi pobieranymi dynamicznie z bazy.
\end{itemize}

\section*{Stan dokumentacji}
Dokumentacja (PDF) została rozbudowana o:
\begin{itemize}
    \item Szczegółowe założenia funkcjonalne (Licznik, Tryb Treningowy, Dziennik).
    \item Koncepcję techniczną (lokalna baza SQLite, sensory sprzętowe).
    \item Wytyczne UX/UI (wysoki kontrast, czytelność w terenie).
\end{itemize}

\end{document}